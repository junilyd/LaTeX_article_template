\documentclass[journal,trans]{IEEEtran}
\usepackage[utf8]{inputenc}
%-----------------------------------------------
%                 A
%                 |
%                 |
%%         ORIGINAL CONTENT                 
%%
%-----------------------------------------------
% ADDITIONAL CONTENT (not in IEEEtran template)
%-----------------------------------------------
\usepackage{xfrac}
\usepackage[font=footnotesize,justification=justified,singlelinecheck=false, labelfont=bf]{caption} %% THIS SETUP IS NOT IEEE STANDARD
\usepackage{pgfplots}
\usepackage{tikz}
\usepackage{bm}
\usepackage{cleveref}
\usepackage{listings}
%-----------------------------------------------
%         ORIGINAL CONTENT
%                 |
%                 |
%                 V
%
% *** CITATION PACKAGES ***
%
\usepackage{cite}

% *** GRAPHICS RELATED PACKAGES ***
%
\ifCLASSINFOpdf
\usepackage{graphicx}% removed [pdftex], due to clash with tikz
  \graphicspath{{img/}{img/tikz/}}
  \DeclareGraphicsExtensions{.pdf,.jpg,.jpeg,.png,.tikz}
\else
  \usepackage[dvips]{graphicx}
  \graphicspath{{img/}}
  \DeclareGraphicsExtensions{.eps}
\fi

% *** MATH PACKAGES ***
%
\usepackage[cmex10]{amsmath}

% *** SPECIALIZED LIST PACKAGES ***
%
\usepackage{algorithmic}

% *** ALIGNMENT PACKAGES ***
%
\usepackage{array}

% *** SUBFIGURE PACKAGES ***
%
\ifCLASSOPTIONcompsoc
 \usepackage[caption=false,font=normalsize,labelfont=sf,textfont=sf]{subfig}
\else
 \usepackage[caption=false,font=footnotesize]{subfig}
\fi

% *** FLOAT PACKAGES ***
%
% \usepackage{fixltx2e}
% \usepackage{stfloats}
% Do not attempt to use stfloats with fixltx2e as they are incompatible.
% Instead, use Morten Hogholm'a dblfloatfix which combines the features
% of both fixltx2e and stfloats:
%
\usepackage{dblfloatfix}

% List of Figures at the end, if invoked
\ifCLASSOPTIONcaptionsoff
 \usepackage[nomarkers]{endfloat}
\let\MYoriglatexcaption\caption
\renewcommand{\caption}[2][\relax]{\MYoriglatexcaption[#2]{#2}}
\fi
% Some IEEE journals/societies require that
% submissions have lists of figures/tables at the end of the paper and that
% figures/tables without any captions are placed on a page by themselves at
% the end of the document. If needed, the draftcls IEEEtran class option or
% \CLASSINPUTbaselinestretch interface can be used to increase the line
% spacing as well. Be sure and use the nomarkers option of endfloat to
% prevent endfloat from "marking" where the figures would have been placed
% in the text.
% the subcaptions:
% For subfig.sty:
% \let\MYorigsubfloat\subfloat
% \renewcommand{\subfloat}[2][\relax]{\MYorigsubfloat[]{#2}}
% However, the above trick will not work if both optional arguments of
% the \subfloat command are used. Furthermore, there needs to be a
% description of each subfigure *somewhere* and endfloat does not add
% subfigure captions to its list of figures. Thus, the best approach is to
% avoid the use of subfigure captions (many IEEE journals avoid them anyway)
% and instead reference/explain all the subfigures within the main caption.
% The latest version of endfloat.sty and its documentation can obtained at:
% http://www.ctan.org/tex-archive/macros/latex/contrib/endfloat/
%
% The IEEEtran \ifCLASSOPTIONcaptionsoff conditional can also be used
% later in the document, say, to conditionally put the References on a 
% page by themselves.




% *** PDF, URL AND HYPERLINK PACKAGES ***
%
\usepackage{url}
% Basically, \url{my_url_here}.


% *** Do not adjust lengths that control margins, column widths, etc. ***
% *** Do not use packages that alter fonts (such as pslatex).         ***
% There should be no need to do such things with IEEEtran.cls V1.6 and later.
% (Unless specifically asked to do so by the journal or conference you plan
% to submit to, of course. )


% correct bad hyphenation here
\hyphenation{op-tical net-works semi-conduc-tor}
%-----------------------------------------------
%                 A
%                 |
%                 |
%%         ORIGINAL CONTENT                 
%%
%----------------------------------------------------
% ADDITIONAL CONTENT SETUP FOR PGFPLOTS and TikZ
%----------------------------------------------------
\usetikzlibrary{shapes,arrows,positioning,patterns,decorations.markings,decorations.pathmorphing}
\usetikzlibrary{external} 
\tikzexternalize[prefix=img/tikz/]
\tikzset{
    >=latex,
    font=\footnotesize,
}
\pgfplotsset{
    compat=newest,
    every axis plot/.append style = {
        very thick,
    },
    every axis legend/.append style={
        font={\scriptsize},
    },
    every axis title/.append style={
        font={\small},
    },
    every axis/.append style={
        legend style={draw=none},
        legend cell align=left,
        axis x line=bottom,
        axis y line=left,
        scaled ticks=false,
        at={(0.8,1)},
        % xticklabel = \pgfmathparse{\tick*1}\sisetup{scientific-notation = engineering}\num{\pgfmathresult},
        % yticklabel = \pgfmathparse{\tick*1}\sisetup{scientific-notation = engineering}\num{\pgfmathresult},
        width=9cm,
        height=4cm,
        axis line style=-latex,
        tick scale binop=\times,
        /pgf/number format/.cd,
        set thousands separator={}],
        set decimal separator={,},
    },
    every axis x label/.style={
        at={(ticklabel* cs:0.5)},
        anchor=east,
        below=0.15,
    },
    every axis y label/.style={
        at={(ticklabel* cs:1.00)},
        anchor=north,
        above=0.05,
    },
    every axis z label/.style={
        at={(ticklabel* cs:1.00)},
        anchor=south,
    },
}

