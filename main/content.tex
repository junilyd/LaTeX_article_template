\section{Content}
In this paper, the syntax and typesetting is considered, for building a paper with \LaTeX. By splitting this into sub-files it is easy to work with the project and flexible to change the content without a change of the structure. However, it might be easier to have the structure in another way, once the project evolves. In the following, each example will be defined in the subsection to this \texttt{content} section. Firstly, a short overview of the structure will be given followed by some tests which is carried out as examples of how it is possible to write a descent scientific paper.
%
\subsection{File Structure}
Since the template is divided into sub-files, these serves the following purpose. The root directory contains a Makefile, master file and two MATLAB test files. The master file contains all inputs to paper and also the ``Title'' and ``Author'' definitions, directly. The root dir also contains the following directories.
%
\input img/Graph_Plot.tex
\subsubsection{Main Directory}
\label{sec:maindir}
The directory \texttt{main} has been created to maintain the structure of a standardized paper. It contains the main files which are defined as inputs in the master file \texttt{master.tex}. This way it should be an easy ``starter'' for initializing the project and one can decide whether to input into these files or directly into the master file, when adding content.
%
\subsubsection{Set Directory}
\label{sec:setdir}
The directory \texttt{set} contains a macro-file and the preamble. The macro file contains the desired syntactic changes to make things easier along the path of writing this paper. This IEEEtran-template\cite{IEEEhowto:IEEEtranpage} can be used for several different types of IEEE-papers, this can be changed by setting the document class in the preamble. The greatest change in the preamble has been implementing \texttt{Tikz}, and defining a \texttt{pgfplotssetup} for matching the template.
%
\subsubsection{Bibtex Directory}
\label{sec:bibtexdir}
The \texttt{bibtex} directory contains the necessary files for citations. The \texttt{sources.bib} is the place to add new litterature sources. Note that it might be necessary to run ``make clean'' followed by ``bibtex master.tex'' a few times, if it is not compiling as expected.
%
\subsubsection{Img Directory}
\label{sec:imgdir}
The \texttt{img} directory contains all figures. The figure format are defined in the preamble, from line 30. This is also the directory where Tikz-files are placed. Each Tikz-file is usually followed by a tex-wrapper with a similar name. This allows for simply applying an input of the wrapper into a file, when inserting a figure into the paper. This process is also automated by utilizing \texttt{matlab2tikz.m}. This is further elaborated in \cref{sec:figures}. This directory contains another directory called \texttt{tikz} this is used as a ``cache'' for pre-compiled tikz figures, in order to reduce compile time. This makes it necessary to remove the \texttt{img/tikz/*} content in order to insert/compile new tikz figures.
%
\subsection{Examples}
\label{sec:examples}
%
\subsubsection{Figures}
\label{sec:figures}.
The handling of figures by means of plotting can be done in the usual way by importing the specified formats as figures inside the latex file. However, the approach in this template is to utilize tikz and pgfplots for plotting. Furthermore, a wrapper is attached to this template for utilizing the MATLAB library called \texttt{matlab2tikz}. \texttt{matlab2tikz} has the ability to convert a MATLAB figure into a tikz-file \cite{m2tikz}. It is possible to create the \LaTeX figure environment in a tex-file which inputs the generated tikz-figure. This can be done in one line from MATLAB, by using the file called \texttt{fig2tikz}. \texttt{fig2tikz} creates a tikz-file and a tex-file with similar names. The tex-file is the only file that that needs to be input into \LaTeX by simply typing \texttt{\\input img/latexfile.tex}. A demo file which shows this procedure is attached to this project. The demo file is named \texttt{demofile\_m2t.m} and is placed in the root directory along with \texttt{fig2tikz.m}. All the figures in this paper is created by \texttt{demofile\_m2t.m}. Note that \texttt{matlab2tikz} has to be installed correctly for this procedure to work \cite{m2tikz}. 
\input img/Sub_Plot
\input img/Surf_Plot
\subsubsection{Math}
\label{sec:math}
The handling of math will be tested by applying a random example in the following.
Consider the linear combination $\veca'=\rowvec{1}{0}{\cdots}{0}$, Since
\begin{equation}
    \veca'\vecX = \rowvec{1}{0}{\cdots}{0}\colvec{X_1}{X_2 }{\vdots}{X_4} = X_1
\end{equation}
and
\begin{equation}
    \veca'\vecmu = \rowvec{1}{0}{\cdots}{0}\colvec{\mu_1}{\mu_2 }{\vdots}{\mu_4} = \mu_1
\end{equation}
\begin{equation}
    \veca'\mtrxsigma\veca = \rowvec{1}{0}{\cdots}{0}
    \begin{bmatrix}
        \sigma_{1n} & \sigma_{2n} & \cdots & \sigma_{n} \\
        \sigma_{12} & \sigma_{22} & \cdots & \sigma_{2n} \\
        \cdots & \cdots & \ddots & \cdots \\
        \sigma_{1n} & \sigma_{2n} & \cdots & \sigma_{n} \\
    \end{bmatrix}
    \colvec{\mu_1}{\mu_2 }{\vdots}{\mu_4} = \sigma_{11}
\end{equation}











% %
% % Figure of me
% %
%     \begin{figure}[!t]
%     \includegraphics[width=1.5in]{img/jacobm}
%     \caption{caption}
%     \label{fig:label}
%     \end{figure}
% %
% % SUBFIGURE
% %
% And here is a subfigure
% \begin{figure*}[!t]
% \centering
% \subfloat[Case I]{\includegraphics[width=2.5in]{jacobm}%
% \label{fig_first_case}}
% \hfil
% \subfloat[Case II]{\includegraphics[width=2.5in]{jacobm}%
% \label{fig_second_case}}
% \caption{Simulation results.}
% \label{fig_sim}
% \end{figure*}
% 
% %
% % TABLE
% %
% \begin{table}[!t]
% \renewcommand{\arraystretch}{1.3}
% \caption{An Example of a Table}
% \label{table_example}
% \centering
% \begin{tabular}{|c||c|}
% \hline
% One & Two\\
% \hline
% Three & Four\\
% \hline
% \end{tabular}
% \end{table}
